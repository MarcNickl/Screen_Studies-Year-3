$$
\setlength{\parindent}{4em}
\setlength{\parskip}{1em}

  
\documentclass{article}
\usepackage[utf8]{inputenc}
\usepackage[english]{babel}

\setlength{\parindent}{4em}
\setlength{\parskip}{1em}
\renewcommand{\baselinestretch}{1.5}

\begin{document}




In this essay I am going to compare the non-fiction work of Werner Herzog and Nick Broomfield, both highly distinguished filmmakers. The essay has the goal of getting a greater understanding of how their work has evolved over the span of their long, unique and inspiring careers. I will compare their work in a number of ways. First I'll be looking at how their beginnings might have influenced them as filmmakers. Then I will look at how their style of filming and the means by which they convey truth and reality to the audience has evolved over the years. The final aspect I will look at is how both Directors approach the same subject matter. Here I intend to look at the film [(Aileen: Life and Death of a Serial Killer, 2003)](https://paperpile.com/c/SdLVV0/U8hP/?noauthor=1) by Nick Broomfield and Into the Abyss, (2012) by Werner Herzog. Both films look at the crime and execution of a serial killer, having interviews with the convicts just a few days before they were to die.

Nick Broomfield's journey in film started when he was studying political science at the University of Essex and law at the University College Cardiff. He borrowed a film camera from the rugby club at the University College Cardiff which he then used to make his first film in Liverpool about the problems of slum demolition and the removal of residents to new housing blocks. The film is called Who Cares [(Who Cares, 1971)](https://paperpile.com/c/SdLVV0/W0Hr/?noauthor=1).  This led to him joining the National Film and Television School in 1973. This is where he met the cinematographer and his lifelong collaborator Joan Churchill.

\end{document}
$$