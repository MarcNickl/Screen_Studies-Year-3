% Options for packages loaded elsewhere
\PassOptionsToPackage{unicode}{hyperref}
\PassOptionsToPackage{hyphens}{url}
%
\documentclass[
]{book}
\usepackage{amsmath,amssymb}
\usepackage{lmodern}
\usepackage{iftex}
\ifPDFTeX
  \usepackage[T1]{fontenc}
  \usepackage[utf8]{inputenc}
  \usepackage{textcomp} % provide euro and other symbols
\else % if luatex or xetex
  \usepackage{unicode-math}
  \defaultfontfeatures{Scale=MatchLowercase}
  \defaultfontfeatures[\rmfamily]{Ligatures=TeX,Scale=1}
\fi
% Use upquote if available, for straight quotes in verbatim environments
\IfFileExists{upquote.sty}{\usepackage{upquote}}{}
\IfFileExists{microtype.sty}{% use microtype if available
  \usepackage[]{microtype}
  \UseMicrotypeSet[protrusion]{basicmath} % disable protrusion for tt fonts
}{}
\makeatletter
\@ifundefined{KOMAClassName}{% if non-KOMA class
  \IfFileExists{parskip.sty}{%
    \usepackage{parskip}
  }{% else
    \setlength{\parindent}{0pt}
    \setlength{\parskip}{6pt plus 2pt minus 1pt}}
}{% if KOMA class
  \KOMAoptions{parskip=half}}
\makeatother
\usepackage{xcolor}
\IfFileExists{xurl.sty}{\usepackage{xurl}}{} % add URL line breaks if available
\IfFileExists{bookmark.sty}{\usepackage{bookmark}}{\usepackage{hyperref}}
\hypersetup{
  hidelinks,
  pdfcreator={LaTeX via pandoc}}
\urlstyle{same} % disable monospaced font for URLs
\setlength{\emergencystretch}{3em} % prevent overfull lines
\providecommand{\tightlist}{%
  \setlength{\itemsep}{0pt}\setlength{\parskip}{0pt}}
\setcounter{secnumdepth}{-\maxdimen} % remove section numbering
\ifLuaTeX
  \usepackage{selnolig}  % disable illegal ligatures
\fi

\author{}
\date{}

\begin{document}
\frontmatter

\mainmatter
\hypertarget{nick-broomfield-werner-herzog}{%
\section{Nick Broomfield \& Werner Herzog}\label{nick-broomfield-werner-herzog}}

\hypertarget{marc-nickl}{%
\subsection{Marc Nickl}\label{marc-nickl}}

\hypertarget{nick-broomfield-and-werner-herzog}{%
\section{Nick Broomfield and Werner Herzog}\label{nick-broomfield-and-werner-herzog}}

\hypertarget{compare-the-non-fiction-work-of-any-two-directors-referenced-within-this-unit.-herzog-v.-broomfield}{%
\subsection{Compare the non-fiction work of any two directors referenced within this unit. Herzog v. Broomfield~}\label{compare-the-non-fiction-work-of-any-two-directors-referenced-within-this-unit.-herzog-v.-broomfield}}

Marc Nickl

Show the contrast of there work at the outset of their careers~

``An endless struggle with great benefits of satisfaction when you get to the end'' (Broomfield, 2017)

\# Introduction

\hypertarget{in-this-essay-i-am-going-to-compare-the-non-fiction-work-of-werner-herzog-and-nick-broomfield-both-highly-distinguished-filmmakers.-the-essay-has-the-goal-of-getting-a-greater-understanding-of-how-their-work-has-evolved-over-the-span-of-their-long-unique-and-inspiring-careers.-i-will-compare-their-work-in-a-number-of-ways.-first-ill-be-looking-at-how-their-beginnings-might-have-influenced-them-as-filmmakers.-then-i-will-look-at-how-their-style-of-filming-and-the-means-by-which-they-convey-truth-and-reality-to-the-audience-has-evolved-over-the-years.-the-final-aspect-i-will-look-at-is-how-both-directors-approach-the-same-subject-matter.-here-i-intend-to-look-at-the-film-aileen-life-and-death-of-a-serial-killer-2003-by-nick-broomfield-and-into-the-abyss-2012-by-werner-herzog.-both-films-look-at-the-crime-and-execution-of-a-serial-killer-having-interviews-with-the-convicts-just-a-few-days-before-they-were-to-die.}{%
\subparagraph{\texorpdfstring{In this essay I am going to compare the non-fiction work of Werner Herzog and Nick Broomfield, both highly distinguished filmmakers. The essay has the goal of getting a greater understanding of how their work has evolved over the span of their long, unique and inspiring careers. I will compare their work in a number of ways. First I'll be looking at how their beginnings might have influenced them as filmmakers. Then I will look at how their style of filming and the means by which they convey truth and reality to the audience has evolved over the years. The final aspect I will look at is how both Directors approach the same subject matter. Here I intend to look at the film \href{https://paperpile.com/c/SdLVV0/U8hP/?noauthor=1}{(Aileen: Life and Death of a Serial Killer, 2003)} by Nick Broomfield and Into the Abyss, (2012) by Werner Herzog. Both films look at the crime and execution of a serial killer, having interviews with the convicts just a few days before they were to die.}{In this essay I am going to compare the non-fiction work of Werner Herzog and Nick Broomfield, both highly distinguished filmmakers. The essay has the goal of getting a greater understanding of how their work has evolved over the span of their long, unique and inspiring careers. I will compare their work in a number of ways. First I'll be looking at how their beginnings might have influenced them as filmmakers. Then I will look at how their style of filming and the means by which they convey truth and reality to the audience has evolved over the years. The final aspect I will look at is how both Directors approach the same subject matter. Here I intend to look at the film (Aileen: Life and Death of a Serial Killer, 2003) by Nick Broomfield and Into the Abyss, (2012) by Werner Herzog. Both films look at the crime and execution of a serial killer, having interviews with the convicts just a few days before they were to die.}}\label{in-this-essay-i-am-going-to-compare-the-non-fiction-work-of-werner-herzog-and-nick-broomfield-both-highly-distinguished-filmmakers.-the-essay-has-the-goal-of-getting-a-greater-understanding-of-how-their-work-has-evolved-over-the-span-of-their-long-unique-and-inspiring-careers.-i-will-compare-their-work-in-a-number-of-ways.-first-ill-be-looking-at-how-their-beginnings-might-have-influenced-them-as-filmmakers.-then-i-will-look-at-how-their-style-of-filming-and-the-means-by-which-they-convey-truth-and-reality-to-the-audience-has-evolved-over-the-years.-the-final-aspect-i-will-look-at-is-how-both-directors-approach-the-same-subject-matter.-here-i-intend-to-look-at-the-film-aileen-life-and-death-of-a-serial-killer-2003-by-nick-broomfield-and-into-the-abyss-2012-by-werner-herzog.-both-films-look-at-the-crime-and-execution-of-a-serial-killer-having-interviews-with-the-convicts-just-a-few-days-before-they-were-to-die.}}

Nick Broomfield's journey in film started when he was studying political science at the University of Essex and law at the University College Cardiff. He borrowed a film camera from the rugby club at the University College Cardiff which he then used to make his first film in Liverpool about the problems of slum demolition and the removal of residents to new housing blocks.~ The film is called Who Cares \href{https://paperpile.com/c/SdLVV0/W0Hr/?noauthor=1}{(Who Cares, 1971)}.~ This led to him joining the National Film and Television School in 1973. This is where he met the cinematographer and his lifelong collaborator Joan Churchill.

Ten years earlier Werner Herzog had started his film career by reading ``an encyclopedia, the fifteen or so pages on filmmaking. Everything I needed to get myself started came from this book.'' \href{https://paperpile.com/c/SdLVV0/n5mQ/?locator=28}{(Cronin and Herzog, 2003:28)} This led to him stealing a 35mm film camera from the Munich Film School, which made it possible for him to make his first film (Herakles, 1962) about competitive bodybuilders. Two year later he started work on the film Spiel im Sand, 1964, however it was never released as ``things were moving out of control'' (Cronin and Herzog, 2003). He continued to make short documentaries till 1970 when he made The Flying Doctors of East Africa.

Being self taught, Herzog had a unique and personal style and the artform of his work has always stood out as being very individual. Throughout the span of Herzog's career, his sensibilities have not changed, although you can see improvements in his film craft.~

In the careers of both Herzog and Broomfield, you see the greatest differences in directorial style in the first films they did, the films are polar opposites in the way they are shot and cut together.~ Who Cares, (1971) is a more personal film in many ways, following a subject matter that Nick Broomfield cared about, taking over a~ year to cut. ``I had a strong feeling making this film that modernity has destroyed so much of people's sense of belonging'' \href{https://paperpile.com/c/SdLVV0/R1mx}{(Who Cares - Nick Broomfield's Official Website, s.d.)} This is clearly seen in the film where each shot, each cut seems purposeful, holding shots where needed and creating a steady decisive look. Generally it feels more mature for a first film.

Compare this to the film Herakles, which has closer ties to earlier montage cinema, inter-cutting between bodybuilders preparing for competition with stock footage of disaster, like the crash at Le Mans in 1955, or post world war two clean up. This creates a faster cutting pace and it seems like a less thought out cut. In Herzog's own words, ``My most immediate and radical lesson came from what was my first blunder, Herakles'' \href{https://paperpile.com/c/SdLVV0/n5mQ/?locator=24}{(Cronin and Herzog, 2003:24)}.

When talking about the style of a filmmaker an important question is. - Where did the style come from and if it evolved why?

As Nick Broomfield studied at the National Film School a connection can be made that the style of his early works like Proud to Be British, 1973 or Juvenile Liaison,1976 were hugely influenced by the popular documentary style at the time, cinéma vérité, Therefore by the tutelage of Colin Young, the founding director of the NFTS, who was at the time writing the manifesto-essay titled Observational Cinema, published in Principles of Visual Anthropology \href{https://paperpile.com/c/SdLVV0/uTip/?locator=99}{(Young and Hockings, 1975:99)}. In the essay he states, ``The difference is between TELLING a story and SHOWING us something'' is completely different.~ This is one of the key principles of cinéma vérite, where the film attempts to show reality without the influence of the storyteller. He also states that, ``\ldots we are spending time perfecting our ability to see straight, shoot straight, and edit the sections so that the parts of our film have a sense of the whole,\ldots{}''\href{https://paperpile.com/c/SdLVV0/uTip/?locator=112}{(Young and Hockings, 1975:112)}

Another point regarding the question of nature or nurture in filmmakers is when Nick Broomfield's collaboration with Joan Churchill comes to an end in the late 1980's. There is a clear change in style in Broomfield's work. His style of documentary changed from the aforementioned Cinéma Vérité to his pioneering on-screen appearance with Participatory Documentary in Driving Me Crazy (1988). This was due to the production of Driving Me Crazy going ``hopelessly out of control.'' \href{https://paperpile.com/c/SdLVV0/XGco}{(Fairweather, 2007)}.~

Fig 1 Nick Broomfield's first appearance on camera

Fig 2 Nick Broomfield on the left

Broomfield made a deal with his producers that he would only continue to work on the film if he could film everything surrounding the production, including the conversations behind the scenes. This gave him the opportunity to experiment.~ In an interview with Aesthetica he said, ``It enables you to establish a more in-depth relationship with people. When filming you show more things than in a feature film, which requires a beginning, middle and end sequence. Appearing in the film gives you the flexibility to include thoughts and flashing of things, contextualising what is happening.'' \href{https://paperpile.com/c/SdLVV0/XGco}{(Fairweather, 2007)} He continues to say ``The thing that excites me about filmmaking is the spontaneity, I want there to be energy, so that it feels real.'' \href{https://paperpile.com/c/SdLVV0/XGco}{(Fairweather, 2007)}

This change in style has continued on till the present day.

``The later films of Nick Broomfield take this notion of constructed truth a stage further as they build themselves around the encounters between subjects and Broomfield's on-screen alter ego -- encounters that, in turn, form the basis for a reflexive dialogue with the spectator on the nature of documentary authenticity.'' \href{https://paperpile.com/c/SdLVV0/oHFX/?locator=11}{(Bruzzi, 2006:11)}

Werner Herzog is more known for his presence on screen and for appearing in front of the camera.~ Similar to Nick Broomfield's first time, he did not appear in front of the camera of his own volition, but because he was ``forced to make an appearance''. \href{https://paperpile.com/c/SdLVV0/n5mQ/?locator=180}{(Cronin and Herzog, 2003:180)} This was due to \href{https://paperpile.com/c/SdLVV0/KRAK}{(Die große Ekstase des Bildschnitzers Steiner, 1974)} being produced in a series called Grenzstationen, where it was stipulated that Herzog had to ``conform to the network's rules, one of which was that the filmmaker had to appear in the film as the chronicler of events'' \href{https://paperpile.com/c/SdLVV0/n5mQ/?locator=180}{(Cronin and Herzog, 2003:180)}.~ Herzog said himself about this, that, ``Some people look at a film like The Great Ecstasy of Woodcarver Steiner and accuse me of self-promotion because I appear in the film.'' ~ However, like with Broomfield, it was due to this experience of appearing in front of camera that he realised the possibilities and potential of this style of filming and it altered his way of working from then on.

Fig 3 Werner Herzog's first appearance on camera

Expand on why it is beneficial for Herzog to appear in front of camera

~There is a difference between operating being in front of camera and using oneself as a character in the story.~

This change in style does not happen in Werner Herzog's career as he has a reputation for saying and doing the same or at least with the same sensibilities.

\hypertarget{poetic-truth-and-direct-cinema}{%
\section{Poetic Truth and Direct Cinema}\label{poetic-truth-and-direct-cinema}}

Both filmmakers have also carved out their own styles in documentary cinema. This can be seen in their `rebellious' alteration from the norms can be seen in both Nick Broomfield's use of actors to recreate scenes as well as Werner Herzog's use of faking interviews.~ Herzog even wrote a script for someone to say in a documentary interview in the guise of poetic truth.~ He justifies his way of working - ``since what moves me has never been reality, but a question that lies behind it {[}beyond; dahinter{]}: the question of truth. Sometimes facts so exceed our expectations---have such an unusual, bizarre power---that they seem unbelievable.'' \href{https://paperpile.com/c/SdLVV0/7OdH/?locator=8-9}{(Herzog and Weigel, 2010:8--9)}

There is a fine line between reality and fiction and both Nick Broomfield and Werner Herzog have teetered on the edge, where the boundary becomes a bit smudged. This creates a dilemma for anyone trying to categorise where a film belongs, it can also be a weird line to go down when trying to make a film where the viewer feels something, whether that be a sense of truth or just simple excitement.~

``By dint of declaration the so-called Cinéma vérité is devoid of vérité'' \href{https://paperpile.com/c/SdLVV0/qSB4}{(Herzog and Ebert, 1999)}, One of the key things that makes Werner Herzog and his work stand out is the originality as he advocates for a more poetic interpretation of truth in documentary filmmaking, sometimes blurring the lines between reality and fiction. He often ``played with the `truth' of the situation to reach a more poetic understanding'' \href{https://paperpile.com/c/SdLVV0/n5mQ/?locator=253}{(Cronin and Herzog, 2003:253)}, this might have come from his self-taught upbringing in film.~

Nick broomfield an alteration to `direct Cinema' in Ghosts and werner herzog with his manifesto on the ecstatic truth~

What films has he exploited poetic truth in?

\hypertarget{broomfield-direct-cinema}{%
\subsubsection{Broomfield Direct Cinema}\label{broomfield-direct-cinema}}

The final alteration in Nick Broomfield's style happened in 2006 when he directed (Ghosts, 2006), which~ is a story about the immigration of a Chinese national to the UK. This was so that she could support her family. She gets an underpaid job in a UK meat~ ~ ~ ~ ~ ~ ~ ~ ~ ~ ~ ~ ~ ~ ~ ~ ~ ~ ~ ~ ~ ~ ~ ~ ~ ~ ~ ~ ~ ~ ~ packing factory. The film recreates the Morecambe Bay cockling disaster where she and 22 other illegal workers drowned picking cockles \href{https://paperpile.com/c/SdLVV0/1IqG}{(Bradshaw, 2007)}.

Ghosts closely follows the style of what he calls ``Direct Cinema'', meaning that he used actors to recreate the story.

There is a lot of noise and opposing opinions on what the term `Direct Cinema' means. The `more common' academic definitions are slightly different from Broomfield's interpretation. The academic difference between Cinéma Vérité and Direct Cinema is that ``Cinéma vérité wanted to explain the raison d'être of life, whereas Direct Cinema wanted to let life reveal itself'' \href{https://paperpile.com/c/SdLVV0/ojqC}{(McLane, 2012)}. However, Broomfield ``calls ``direct cinema'' also known as ``enhanced reality'', where non-actors play themselves in scripted dialogue.'' \href{https://paperpile.com/c/SdLVV0/PkHs}{(Williams, 2017)} This definition better covers what Broomfield has stylistically done in Ghosts (2006) and Battle for Haditha (2007) where he used actors to recreate scenes that would otherwise not have been shot.

Whilst this style of filming is on the edge of what I would consider a documentary, due to its use of non-actors, I do think this is where Broomfield and Herzog are closest stylistically. Mention film by Herzog (The Act of Killing)

In an article by Candis Callison Comparing Direct Cinema and Cinema Verité

Merge the Two Sections on truth~

To merge the two categories use Herzog being the executive producer on as a transition point

In the film The Act of Killing

Herzog was an executive producer in The Act of Killing

The documentary, although not Herzog's direct work, is the~

One more para on stylistic interpretation of truth and how both herzog and broomfield have gone to the extremes of pushing what is considered real to the limits~

\hypertarget{comparing-two-films}{%
\section{Comparing two films}\label{comparing-two-films}}

To gain a closer insight into the styles of two filmmakers, one needs to look at their production of films having the same subject matter.~ This is rarely possible, however, in the case of Nick Broomfield and Werner Herzog, it is possible.~ Both of them made films about murderers who were on death row, including interviews with the convicts in their last days.~ Aileen: Life and Death of a Serial Killer (2003) by Nick Broomfield and Into the Abyss (2011) by Werner Herzog.

Aileen: Life and Death of a Serial Killer (2003), is the second part of the Aileen Wuornos story. The first part happened 10 years earlier with Aileen Wuornos: The Selling of a Serial Killer (1992).

There have been many documentary and fiction films looking at the story of Aileen Wuornos, like the film by Patty Jenkins called Monster (2003). In fact, in her research for the role of Aileen, Charlize Theron, who was to play that role, asked Nick Broomfield, who at the time was shooting the second Aileen Wuornos documentary film, for footage which she could use as character reference. Nick Broomfield sent `` them a rough cut of the second film'' \href{https://paperpile.com/c/SdLVV0/QZbN/?locator=62}{(Wood, 2005:62)}.~

In Nick Broomfield's autobiography he was asked by Jason Wood, ``Nearly all the articles on Monster deflected attention back to your films'' to which he responded with~

``I had the choice of either working with them and trying to make it into a positive experience that would help both films and an understanding of Aileen {[}or not{]}\ldots{} I think we actually did benefit from each other and, in a way, the films authenticated Charlize's performance. The films fed off each other.''\href{https://paperpile.com/c/SdLVV0/QZbN/?locator=62-63}{(Wood, 2005:62--63)}

In spite of being a documentary, the critics have stated that the film by Nick Broomfield had more drama than the fictional work of Patty Jenkins.~ This goes to show the power of a good story when brought to life in a documentary well made.~ ``In the end, Broomfield and Churchill's search for truth inevitably trumps Jenkins's fictionalisation. Even Theron's remarkable acting\ldots. is superseded by Broomfield's interview with Wuornos on the eve of her execution.'' \href{https://paperpile.com/c/SdLVV0/veXD}{(Patterson, 2004)}

\hypertarget{herzog-bit-on-death-row}{%
\subsection{Herzog Bit on death row}\label{herzog-bit-on-death-row}}

Herzog's film on the life of a person in death row is~

Into The Abyss: A Tale of Death, a Tale Of Life (2012) is a film by Werner Herzog following the life and death of Michael Perry who got put on death row for the murder of three people.

Into the abyss only had 4 hours of shoot footage - this is seen in the film~

Sensitivity when filming characters~

``offering an account of the convicted's humanity'' \href{https://paperpile.com/c/SdLVV0/wycg/?locator=245}{(Picart et al., 2016:245)}

, \{often for those living on death row documentary on Aileen Wuornos by Nick Broomfield and Werner Herzog's Into the

Abyss, 2011, in a society obsessed with victim culture.''\} \href{https://paperpile.com/c/SdLVV0/wycg/?locator=245}{(Picart et al., 2016:245)}

Was Nick Broomfield the first filmmaker to cover Aileen Wuornos

When~

Into The Abyss: A Tale of Death, a Tale Of Life (2011){[}2MN7{]}~

Short story behind Into the abyss

(Biles, 2012)

Through both films God is referenced a lot, this could either be a point to look at however I don't know how both filmmakers react to it

\hypertarget{conclusion-to-the-direct-comparison}{%
\subsection{Conclusion to the direct comparison~}\label{conclusion-to-the-direct-comparison}}

In their works on death row both Werner Herzog and Nick Broomfield have been described as ``offering an account of the convicted humanity'' \href{https://paperpile.com/c/SdLVV0/wycg/?locator=245}{(Picart et al.~2016, page 245)} in Framing Law and Crime: An Interdisciplinary Anthology. This makes watching the films even more emotionally taxing as you begin to relate to the characters portrayed in the film.

~~The two filmmakers also have a certain level of Sensitivity and humanity when filming characters although disagreeing with their actions treating them like humans.

~~This also means there is clear objection with the death penalty and the documentary arnt aftaid to hide there authers views. this goes against conventional wisdom of observational cinema.

~~is a side the documentary ~ allows a clear objection towards the death penalty to be visible in both documentary. this is against~

\hypertarget{their-influences-on-film}{%
\subsection{Their influences on film/~}\label{their-influences-on-film}}

how they've affected the filmmaking scene {[}2MN8{]} 100 words

Short paragraph on how they have influenced the filmmaking community

``Someone I've looked up to for many years, someone who represented sort of a gold standard for the kind of documentary making I aspired to do'' \href{https://paperpile.com/c/SdLVV0/YJfg}{(Sheffield Doc Fest 2017 The BBC Interview Louis Theroux meets Nick Broomfield - YouTube, s.d.)}

(Sheffield Doc Fest 2017 The BBC Interview Louis Theroux meets Nick Broomfield)

\hypertarget{conclusion}{%
\subsection{Conclusion}\label{conclusion}}

Throughout their career in film, Herzog and Broomfield have used different stylistic interpretations of documentary cinema to tell a story, together they have gone from one end of the documentary spectrum to the other. They have captured `real life' as Dr.~Jeffrey Geiger referred to in a speech ``Both men's films show that real life occurs in the margins'' \href{https://paperpile.com/c/SdLVV0/kmPU}{(Geiger, 2006)}. Both filmmakers are trying and also succeeding to bring the human condition to the screen, often choosing subjects who live on the outskirts of society, on the edge of social order. They have been highly inspirational to a whole generation of filmmakers and have affected the world as we know it today, opening the eyes of their viewers and raising awareness.

When a camera is put in someone's face they stop acting naturally and become self conscious.~ In the words of Louis Marcorelles ``I want to discover something about people. When you interview someone they always tell you what they want you to know about them'' \href{https://paperpile.com/c/SdLVV0/Y49G/?locator=55}{(Marcorelles, 1973:55)} and ``You don't show the whole of a subject; you select; and your selection matters'' \href{https://paperpile.com/c/SdLVV0/Y49G/?locator=53}{(Marcorelles, 1973:53)} Being able to direct and manipulate the footage to get some semblance of what `real' life looks like requires great skill and practice. Both filmmakers have got the ability to do exactly that.

To conclude both filmmakers are highly distinguished and respected throughout the field of filmmaking. Both peoples' films have made the world a better place. So, yes, both filmmakers are different in a myriad of ways, but they still share the same values in making films.

\hypertarget{bibliography}{%
\section{Bibliography}\label{bibliography}}

\backmatter
\end{document}
